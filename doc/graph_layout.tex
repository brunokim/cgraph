\section{\texttt{graph\_layout}}

\subsection{Types}

\subsubsection{\texttt{coord\_t}}

Euclidean coordinates in 2D.

\subsubsection{\texttt{box\_t}}

Box (rectangle) definition in 2D, given by its SW and NE vertices in a 
positively oriented world frame, such as the screen. Images may have
a negatively oriented frame, with $y$ pointing down.
It is necessary that \texttt{box.sw.y < box.ne.y} and \texttt{box.sw.x < box.ne.x}.

\subsubsection{\texttt{color\_t}}

Array with 4 colors between 0 and 255, inclusive: red ($R$), green ($G$), blue ($B$) and 
alpha ($A$). $A=0$ means totally transparent, and $A = 255$ means totally opaque.

\subsubsection{\texttt{circle\_style\_t}}

SVG circle style.

\begin{description}
 \item[\texttt{radius}] Circle radius in pixels.
 \item[\texttt{width}] Stroke width in pixels. This is added to the radius for total
 size.
 \item[\texttt{fill}] Color of the fill.
 \item[\texttt{stroke}] Color of the stroke.
\end{description}

\subsubsection{\texttt{path\_style\_t}}

SVG path style.

\begin{description}
 \item[\texttt{type}] Path type.
 \item[\texttt{from, to}] Path origin and destination.
 \item[\texttt{control}] Control point
 \item[\texttt{width}] Stroke width in pixels.
 \item[\texttt{color}] Stroke color.
\end{description}

 For \texttt{style.type == GRAPH\_STRAIGHT}, draws a straight line from origin
 to destination.
 
 For \texttt{style.type == GRAPH\_PARABOLA}, draws a parabola from origin
 to destination using the control point.
 
 For \texttt{style.type == GRAPH\_CIRCULAR}, draws the arc of a circle from 
 origin to destination using the control point as the circle center.

\subsection{Layout}

\subsubsection{\texttt{graph\_layout\_random}}

Place points uniformly inside specified box.

\begin{description}
 \item[Preconditions]~\\
   \texttt{box} must be a valid box.\\
   \texttt{p} must have dimension $n$.
 \item[Postconditions] \texttt{p[i]} is a random coordinate inside \texttt{box}.
\end{description}

\subsubsection{\texttt{graph\_layout\_random\_wout\_overlap}}

Place points with specified radius uniformly avoiding overlap 
with probability $t$.

\begin{description}
 \item[Preconditions]~\\
   \texttt{radius} must be positive.\\
   $t$ must be a valid probability ($0 \ge t \ge 1$).\\
   \texttt{p} must have dimension $n$.
 \item[Postconditions] \texttt{p[i]} is a random coordinate.
\end{description}

The algorithm determines a box with size $l$ such that, if $n$ 
points with radius $r$ are thrown within it, will not have any 
collision with probability $t$. The formula is derived in 
\href{http://math.stackexchange.com/q/325844/37667}{Math Exchange}.

\begin{equation*}
 l = \frac{nr}{2} \sqrt{\frac{2 \pi}{-\log(1-t)}}
\end{equation*}

\subsubsection{\texttt{graph\_layout\_circle}}

Place points with specified radius in a circle without overlap.

\begin{description}
 \item[Preconditions]~\\
   \texttt{radius} must be positive.\\
   \texttt{p} must have dimension $n$.
 \item[Postconditions] \texttt{p[i]} is a coordinate in a circle.
 \item[Return value] Circle bounding box size.
\end{description}

Points are positioned sequentially in a circle, starting from the rightmost
and following in counterclockwise order.

\subsubsection{\texttt{graph\_layout\_circle\_edges}}

Fill edge style for a circular layout.

\begin{description}
 \item[Preconditions]~\\
   \texttt{size} must be the circle bounding box size.\\
   \texttt{width} must be positive.\\
   \texttt{color} must be a valid color.\\
   \texttt{es} must have dimension $m$.\\
   \texttt{edge\_style} must have dimension 2.
 \item[Postconditions]~\\
   \texttt{es[i]} is one of the styles \texttt{CIRCULAR} or \texttt{PARABOLA}.\\
   \texttt{edge\_style[0]} is the \texttt{CIRCULAR} style.\\
   \texttt{edge\_style[1]} is the \texttt{PARABOLA} style.\\
\end{description}

This function maps \texttt{es} to a circular or parabolic style, where
an edge is circular if its endpoints are adjacent in a circle, and 
parabolic otherwise.

\subsubsection{\texttt{graph\_layout\_degree}}

Place points in concentric shells, with highest degrees near the center.

\begin{description}
 \item[Preconditions]~\\
   \texttt{radius} must be positive.\\
   \texttt{p} must have dimension $n$.
 \item[Postconditions] \texttt{p[i]} is a coordinate.
\end{description}

Each shell is attached to a degree value; the inner shell contains elements
of the highest degree, and the outer shell contains elements with the lowest
degree. In each shell, elements are placed equally apart.

\subsection{Printing}

\subsubsection{\texttt{graph\_print\_svg}}

Prints graph as SVG to file, using vertex coordinates given in p and with a 
style for each point and edge.

\begin{description}
 \item[Preconditions]~\\
   \texttt{p} must have dimension $n$.
   \texttt{point\_style} must have dimension $n$.
   \texttt{edge\_style} must have dimension $m$.
 \item[Postconditions]
   \texttt{filename} is a valid SVG file.
\end{description}

Edges are ordered according to vertices' order. In undirected graphs, 
an edge $E_{ij}$ is considered only if $i < j$. In directed graphs,
mutual edges will superimpose if \texttt{edge\_style.type == GRAPH\_STRAIGHT}.

\subsubsection{\texttt{graph\_print\_svg\_one\_style}}

Prints graph as SVG to file, using vertex coordinates given in p and with a 
single style for all points and edges.

\begin{description}
 \item[Preconditions]~\\
   \texttt{p} must have dimension $n$.
 \item[Postconditions]
   \texttt{filename} is a valid SVG file.
\end{description}

The edge style type is ignored, using only \texttt{GRAPH\_STRAIGHT}.

\subsubsection{\texttt{graph\_print\_svg\_some\_styles}}

Prints graph as SVG to file, using vertex coordinates given in \texttt{p} and
with a number of styles given. The mapping vertex$\to$style is given in \texttt{ps},
and the mapping edge$\to$style is given in \texttt{es}.

\begin{description}
 \item[Preconditions]~\\
   \texttt{p} must have dimension $n$.\\
   \texttt{ps} must have dimension $n$.\\
   \texttt{es} must have dimension $m$.\\
   \texttt{point\_style} must have dimension \texttt{num\_point\_style}.\\
   \texttt{edge\_style} must have dimension \texttt{num\_edge\_style}.
 \item[Postconditions]
   \texttt{filename} is a valid SVG file.
\end{description}

This function tries to avoid extensive memory utilization one just some 
styles are desired. If vertex $v_i$ should have style $S_j$, then 
\texttt{ps[i] = j}. Ditto for edges.

Edge order is based on vertices order. In undirected edges, edge $E_{ij}$ 
is considered only if $i < j$.
