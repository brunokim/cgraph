\section{\texttt{set}}

This module provides a structure able to efficiently include, remove and query integers in
a hash table. Its possible to iterate over all elements via a linked list.

 This data structure automatically grows to store more integers efficiently, but will not
shrink if items are removed, wasting memory. Consider using \lstinline!set_copy! to copy
the structure using less memory.

\subsection{Constants}

The following constants can be redefined only during compilation.

\begin{table}[!hb]
 \begin{tabular}{|llr|}
  \hline
  Constant                         & Value & Description \\ \hline
  \lstinline!SET_UTILIZATION_RATE! & 0.75  & Maximum utilization rate of hash table. \\
  \hline
 \end{tabular}
\end{table}

\subsection{Types}

\begin{lstlisting}
 typedef struct set_t set_t;
 typedef struct set_entry_t set_entry_t;
 struct set_entry_t {
   int key;
   set_entry_t *next;
 }
\end{lstlisting}

A set is an object of the type \lstinline!set_t!, which is basically an array of \lstinline!set_entry_t!.
An entry contains a key and a pointer to a next element, allowing to traverse all entries as a linked
list terminated by a \NULL. The head can be fetched with \lstinline!set_head(set_t *set)!.

\subsection{Allocation and deallocation}

\begin{lstlisting}
 set_t *new_set(int minimum);
 void  delete_set(set_t *set);
\end{lstlisting}

A set is created via \lstinline!new_set!, where \lstinline!minimum! is the expected number of
elements to be inserted. It preallocates a table large enough to contain this number of
elements, accounting for utilization rate. This may be interesting to avoid multiple memory
allocations if the table needs to grow. 
\lstinline!delete_set! deallocates memory requested for the data structure.

\subsection{Insertion and retrieval}

\begin{lstlisting}
 error_t set_put(set_t *set, int v);
 
 bool set_contains(const set_t *set, int v);
 int set_get(const set_t *set, int pos);
 int set_index(const set_t *set, int v);
\end{lstlisting}

\lstinline!set_put(set,v)! inserts a value \lstinline!v! in \lstinline!set!, with $\mathcal{O}(1)$ amortized cost. 
If the utilization rate goes beyond \lstinline!SET_UTILIZATION_RATE!, a table roughly twice bigger is allocated 
and populated with $\mathcal{O}(n)$ operations. If there is no memory available, the function returns \lstinline!ERROR_NO_MEMORY!;
otherwise, returns \lstinline!ERROR_SUCCESS!.

\lstinline!set_contains! checks whether a value is in the given set, with $\mathcal{O}(1)$ amortized cost. 
\lstinline!set_get! returns the value in position \lstinline!pos! in the linked list, and \lstinline!set_index!
returns the position of a value \lstinline!v! in the list, or -1 if there is no such value in the set. 
Both operations have average cost $\mathcal{O}(n/2)$.

Prerequisites: \lstinline!pos! should be between 0 and $n$ for \lstinline!set_get!.

\subsection{Random retrieval}

\begin{lstlisting}
 int set_get_random(const set_t *set);
 int set_get_random_r(const set_t *set, unsigned int *seedp);
\end{lstlisting}

Both functions pick an element from the table with uniform probability. The reentrant version \lstinline!set_get_random_r!
accepts a pointer to a seed that will be passed to \lstinline!rand_r!. The non-thread-safe version is equivalent to
\lstinline!set_get_random_r(set, NULL)!.

The implementation uses two strategies to pick a random element: if the table is almost empty (with $n < 4\sqrt{s}$, where $s$ is
the size of the table), it selects a number $i$ uniformly from $[0,n)$ and picks the $i$-th element at the linked list. Otherwise, 
it selects slots at random until it finds a non-empty slot, and returns it. This approach takes $s/n$ selections in average.
The rationale for this algorithm is explained at annex \ref{annex:random-picking}.

\subsection{Removing}

\begin{lstlisting}
 bool set_remove(set_t *set, int v);
 void set_clean(set_t *set);
\end{lstlisting}

\lstinline!set_remove! removes a given element from the set. If the element is present, the function returns true and the element is 
removed with $\mathcal{O}(n/2)$ operations in average. Otherwise, the function returns false with $\mathcal{O}(1)$ operations.

\lstinline!set_clean! cleans all slots, without freeing any memory.

The table is not shrinked if the utilization rate is low. If it's necessary to free memory, its recommended to copy the set with
\lstinline!set_copy! and delete the current structure.

\subsection{Set operations}

\begin{lstlisting}
 error_t set_union(set_t *dest, const set_t *other);
 void set_difference(set_t *dest, const set_t *other);
 void set_intersection(set_t *dest, const set_t *other);
\end{lstlisting}
 
\lstinline!set_union!, \lstinline!set_difference! and \lstinline!set_intersection! compute, respectively, the union, intersection 
and difference between the two arguments, and store the result mutating the first one. 

If there is no memory available for \lstinline!set_union!, 
it returns \lstinline!ERROR_NO_MEMORY!; otherwise, it returns \lstinline!ERROR_SUCCESS!. Computing the difference and intersection 
does not need additional memory, so both have type \lstinline!void!.

\subsection{Querying}

\begin{lstlisting}
 int set_size(const set_t *set);
 int set_table_size(const set_t *set);
 set_entry_t *set_head(const set_t *set);
\end{lstlisting}

\lstinline!set_size! returns the number of elements inserted into the set, and \lstinline!set_table_size! returns the size of the
table used.

\lstinline!set_head! returns the first element in the linked list of entries. If \lstinline!set_optimize! were never called, the
elements are presented in insertion order. This shouldn't be used to change the content of an entry, which would invalidate set invariants.

\subsection{Structure optimization}

\begin{lstlisting}
 void set_optimize(set_t *set);
\end{lstlisting}

As elements are inserted into a hash table, the linked list can get very tangled jumping from far away points in memory. \lstinline!set_optimize!
try to improve memory locality by rebuilding the linked list in sequential order, thus reducing the amount of cache misses when traversing
it.

\subsection{Copying}

\begin{lstlisting}
 set_t *set_copy(const set_t *set);
 void set_to_array(const set_t *set, int *arr); 
 int* set_to_dynamic_array(const set_t *set, int *n);
\end{lstlisting}

\lstinline!set_copy! creates a new deep copy of the input set, using as much memory as strictly needed. Keys are inserted in the same order as the
original set. If there isn't enough memory, the function returns \NULL.

\lstinline!set_to_array! populates an array with the keys in the set. The array must already be allocated. 

\lstinline!set_to_dynamic_array! allocates an array and populates it with the keys in the set. The \lstinline!n! parameter is optional, holding the
size of the array. If there isn't enough memory, \lstinline!n! receives 0 and the function returns \NULL.

\subsection{Printing}

\begin{lstlisting}
 void set_print(const set_t *set);
 void set_fprint(FILE *stream, const set_t *set);
\end{lstlisting}

 Prints a set to the indicated stream, or the standard stream in \lstinline!set_print!.