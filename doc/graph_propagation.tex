\section{\texttt{graph\_propagation}}

Information dissemination simulation in networks are implemented in
CGraph in a more abstract way, as there is lots in common between 
different propagation models.

Propagation models consists in a state diagram that represent the
transition sequence for each individual, where one of them is 
the \textit{infectious state}. At each time step, an infectious individual 
sends a message to one of its adjacents, chosen from an uniform distribution.
Care should be taken to determine the next state if an individual receives
more than one message per time step.

Models are implemented using two callbacks that are called in each time step: 

\texttt{state\_transition\_f} determine the next state vector (ie, in which 
state each individual is in);

and \texttt{is\_propagation\_end} determines if the propagation has ended. 

Some models may never reach an end, so there's an additional condition that 
each simulation will run for at most $K \log_2 n$ iterations, where $K$ is 
defined in \texttt{GRAPH\_PROPAGATION\_K}. It can be redefined during
compilation with 

\texttt{-DGRAPH\_PROPAGATION\_K=10}

\subsection{Types}

\subsubsection{\texttt{message\_t}}

Message type storing the origin \texttt{orig} and destination \texttt{dest} of
a message.

\subsubsection{\texttt{propagation\_step\_t}}

Structure storing information on a propagation time step: its state vector
and the messages exchanged.

\begin{description}
 \item[\texttt{n}] Number of individuals in this time step.
 \item[\texttt{state}] State vector, where \texttt{state[i]} is the state of 
   individual $i$.
 \item[\texttt{num\_message}] Number of messages exchanged, that must be equal
   to the number of individuals in the infectious state.
 \item[\texttt{message}] Message array, storing the origin and destination of 
   messages.
\end{description}

\subsubsection{\texttt{state\_transition\_f}}

Callback for state transition, implemented by the propagation model.

\begin{description}
 \item[Preconditions]~\\
   \texttt{next} must have dimension $n$.
   \texttt{curr} must be information about the current step, including exchanged
   messages.\\
   \texttt{n} is the number of elements, that in a dynamic network may be 
   different than the one in the current time step.\\
   \texttt{params} is a pointer to model specific parameters.\\
   \texttt{seedp} is a pointer to a PRNG state variable, or \texttt{NULL}.
 \item[Postcondition] \texttt{next[i]} is the next state of the element $i$.
\end{description}

\subsubsection{\texttt{is\_propagation\_end}}

Callback for simulation termination, implemented by the propagation model.

\texttt{state} is the state vector, and \texttt{num\_step} is the current
iteration number. \texttt{params} is a pointer to model specific parameters.

\subsection{Functions}

\subsubsection{\texttt{graph\_count\_state}}

Counts number of individuals in $s$ that are in the given state.

\subsubsection{\texttt{graph\_propagation}}

Simulates a propagation in graph with a given initial state vector using
the given propagation model.

\begin{description}
 \item[Preconditions]~\\
   \texttt{init\_state} is a valid state vector with dimension $n$.\\
   \texttt{model} is a valid propagation model.\\
   \texttt{params} is a pointer to the model specific parameter structure.
 \item[Postcondition]~\\
   \texttt{num\_step} is the number of steps in simulation.
 \item[Return value]~\\
   Array of \texttt{propagation\_step\_t}.
 \item[Memory deallocation]~\\
   \texttt{int num\_step;}\\
   \texttt{propagation\_step\_t *step = graph\_propagation(..., \&num\_step, ...);}\\
   \texttt{delete\_propagation\_steps(step, num\_step);}
\end{description}

There is a reentrant version \texttt{graph\_propagation\_r}, that expects a
pointer to the PRNG state variable, allowing reproducible simulations.

\subsubsection{\texttt{delete\_propagation\_steps}}

Deallocate a \texttt{propagation\_step\_t} array that was allocated with 
\texttt{graph\_propagation}.

\subsubsection{\texttt{graph\_animate\_coefficient}}

Creates animation frames of a propagation in the given graph.

\begin{description}
 \item[Preconditions]~\\
   \texttt{folder} is an existing folder.\\
   \texttt{p} is a coordinate array with dimension $n$.\\
   \texttt{num\_state} is the number of states in the propagation model used.
   \texttt{step} is a propagation step array with dimension \texttt{num\_step}.
 \item[Postcondition]~\\
   The given folder has \texttt{num\_step} SVG files with name format 
   \texttt{frame\%05d}, numbered incrementally from 0.
\end{description}

\subsubsection{\texttt{graph\_propagation\_freq}}

Compute the number of individuals in each state at each propagation step.

\begin{description}
 \item[Preconditions]~\\
   \texttt{step} is an array with dimension \texttt{num\_step}.\\
   \texttt{freq} is an allocated matrix with dimensions \texttt{num\_step}
    $\times$ \texttt{num\_state} \\
   \texttt{num\_state} is the number of states in the propagation model used.
 \item[Postcondition]~\\
   \texttt{freq[i][s]} is the number of individuals in state $s$ at iteration 
    $i$.
\end{description}

\subsection{Models}

\subsubsection{SI}
\subsubsection{SIS}
\subsubsection{SIR}
\subsubsection{SEIR}
\subsubsection{Daley-Kendall}