\section{Annex}

\subsection{Picking an element at random in a hash table}
\label{annex:random-picking}

Let $n$ be the number of elements in the table, and $s$ the size of the table. The utilization rate is given as $r = n/s$.
The probability to choose a non-empty slot uniformly at the first try is $r$; at the second try is $r(1-r)$; at the third try
is $r(1-r)^2$ and so on. The expected number of tries $k$ is 

\begin{align*}
 k &= 1 \cdot r + 2 \cdot r(1-r) + 3 \cdot r(1-r)^2 + \ldots \\
   &= r \sum_{n=1}^\infty n (1-r)^{n-1} \\
   &= r (1/r^2) = 1/r \\
   &= s/n
\end{align*}

If we pick a random number between 0 and $n-1$ and walk through the linked list, we expect to step through $n/2$ elements at average. 
Let $a$ be the time per random try, and $b$ the time of a step. We should use random picking if 

\begin{align*}
 a \cdot s/n &< b \cdot n/2 \\
  (2a/b)s    &< n^2 \\
        n    &> \sqrt{(2a/b)s}
\end{align*}

We expect that $a>b$, and use $a/b = 8$ to derive the rule that $n > 4\sqrt{s}$ to switch between one approach and the other. 
