\section{\texttt{graph\_model}}

\subsection{Graph creation}

These functions creates new graphs, whose memory should be managed by the caller.

The reentrant versions \texttt{new\_erdos\_renyi\_r}, 
\texttt{new\_watts\_strogatz\_r} and \texttt{new\_barabasi\_albert\_r} accept 
a state argument that will be used to call \texttt{rand\_r} for pseudo-random 
number generation. 
Two calls with the same state argument yield the same graph and same final
state, allowing reproducibility.

\subsubsection{\texttt{new\_clique}}

Creates a complete network with $n$ vertices.

\begin{description}
 \item[Preconditions] $n > 0$
 \item[Return value] An undirected, unweighted complete graph, or \texttt{NULL}
 in case of memory exhaustion.
\end{description}

It should be noticed that the data structure is inefficient to represent
large dense graphs, so it is recommended to check for memory exhaustion upon
return.

\subsubsection{\texttt{new\_erdos\_renyi}}

Creates a random network with $n$ vertices and average degree $k$.

\begin{description}
 \item[Preconditions] ~\\
   $n > 0$ \\
   $0 < k < n$
 \item[Return value] An undirected, unweighted random graph.
\end{description}

There is no guarantee that the network will be connected. The size and 
characteristic of the largest component follow different regimes 
depending on $k$:

\begin{table*}[!hb]
 \centering
 \begin{tabular}{lll}
  \hline
  Regime       & Size      & Loop \\ \hline
  $k < 1$      & $\log n$  & No loop\\
  $k = 1$      & $n^{1/2}$ & No loop\\
  $k > 1$      & $n^{2/3}$ & Some loops\\
  $k > \log n$ & $n$       & Many loops \\\hline
 \end{tabular}
\end{table*}

\subsubsection{\texttt{new\_watts\_strogatz}}
\subsubsection{\texttt{new\_barabasi\_albert}}
