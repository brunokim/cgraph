\section{\texttt{graph}}

The \texttt{graph} module has the basic algorithms and data structures to manipulate graphs. All modules starting with \texttt{graph\_*} depends on this module.

The basic data structure is \texttt{graph\_t}, which uses adjacencies lists to store edges. All storage is done in-memory, which means the graph size is limited to 
what will fit with your RAM requirements. Also, as everything is indexed by native \texttt{int}s, storage may increase whether it is stored in a 32-bit or 64-bit system.
Also, in 32-bit systems the maximum number of edges and vertices is $2^{31} \approx 2.2 \times 10^9$.

Several types of graphs are supported, although more complex types require more memory. The basic traits which can be combined are listed below. Simple graphs 
(undirected, unweighted and unlooped, true graph) need approximately $4N+16M$ bytes in 32-bit and $8N + 32M$ bytes in 64-bit systems.

\begin{description}
 \item[Directed] Directed graphs consider edges to be ordered sets, i.e., where order matters. An edge between vertices $V_1$ and $V_2$ is different from an edge between
 vertices $V_2$ and $V_1$. Directed graphs need approximately $4N + 24M$ bytes in 32-bit and $8N + 48M$ bytes in 64-bit systems. 
  Creation flag is \texttt{GRAPH\_DIRECTION}.
 
 \item[Weighted] Weighted graphs attach weights to edges, stored in \texttt{float}s (standardized to be 32-bits). Weighted graphs need additionaly $4M$ bytes of storage.
 Creation flag is \texttt{GRAPH\_WEIGHT}.
 
 \item[Pseudo] Pseudo graphs allow multiple edges between the same vertices, which can be directed or weighted. Pseudo graphs don't need additional storage.
 Creation flag is \texttt{GRAPH\_PSEUDO}.
 
 \item[Looped] Looped graphs allow self-loops, or edges from a vertex to itself. Looped graphs need additional $4N$ bytes in 32-bit and $8N$ bytes in 64-bit systems.
 Creation flag is \texttt{GRAPH\_LOOP}.
\end{description}

Graphs can be converted between types, from a more complex to a simpler. This will be discussed in subsection \ref{sec:graph-conversion}.

\subsection{Allocation and deallocation}
\label{sec:graph-allocation}

\subsubsection{Creation}

\begin{lstlisting}[language=C,frame=single]
graph_t *new_simple_graph();
graph_t *new_graph(unsigned int flags);
\end{lstlisting}

Graphs are created using one of the functions above. \texttt{new\_simple\_graph} creates a simple graph, which is equivalent to \texttt{new\_graph(0)}. To
create more complex graphs its needed to pass flags bitwise OR'ed together. For example, to create a directed, looped, weighted graph you should call
\texttt{new\_graph(GRAPH\_DIRECTION | GRAPH\_LOOP | GRAPH\_WEIGHT)}.

If there isn't enough memory, or if illegal flags are passed, both function return \texttt{NULL}.

\subsubsection{Deallocation}

\begin{lstlisting}[language=C,frame=single]
void *delete_graph(graph_t *graph);
\end{lstlisting}

\subsection{Conversion}
\label{sec:graph-conversion}

These functions convert from a type of graph to a simpler one. For implementation safety, all functions return a new graph, thus possibly creating duplicates in memory.

\subsubsection{Threshold}

\begin{lstlisting}[language=C,frame=single]
graph_t *graph_threshold
 (const graph_t *original, double threshold, 
  bool keep_weights);
\end{lstlisting}

  Removes every edge with weight smaller than the specified threshold. If \texttt{keep\_weights} is false, the resulting graph is unweighted.
  Otherwise, the new graph keeps all weights equal to or bigger than the threshold from the input graph.
  
  To keep all edges removing weights, use \lstinline!graph_threshold(original_graph, -1.0/0.0, false)! (threshold $= -\infty$).
  
  If \lstinline!original! is not weighted, or if memory was exhausted, the function returns \NULL.

\subsubsection{Simmetry}

\begin{lstlisting}[language=C,frame=single]
graph_t *graph_dual(const graph_t *original);
graph_t *graph_simmetry
  (const graph_t *original, bool keep_directed);
graph_t *graph_direct
  (graph_t *original, bool split_weights);
\end{lstlisting}

  The dual graph of a directed graph is the graph with all its edges reversed. The symmetric graph is the union of a graph with its dual, thus 
  converting a directed graph into an undirected one. 
  
  \lstinline!graph_dual! receives a directed graph and returns its directed dual. 
  
  \lstinline!graph_simmetry! receives a directed graph and returns its dual. If \lstinline!keep_directed! is true, the resulting graph is directed, and
  dual edges keep their weights; otherwise, the resulting graph is undirected, and the weights of dual edges are summed together.
  
  \lstinline!graph_direct! receives an undirected graph and returns its directed equivalent, where an edge $E_{ij}$ is split into $E_{i\to j}$ and $E_{j \to i}$.
  If \lstinline!split_weights! is true, its weight is split evenly between the new edges; otherwise, both edges receive the same weight.
  
  If \lstinline!original! isn't of the specified type, or if memory was exhausted, the function returns \NULL.

\subsubsection{Remove self loops}

\begin{lstlisting}[language=C,frame=single]
  graph_t *graph_remove_self_loops(const graph_t *original);
\end{lstlisting}

  Remove self loops from the original looped graph, returning an unlooped graph.
  
  If \lstinline!original! isn't looped, or if memory was exhausted, the function returns \NULL.

\subsubsection{Coalesce}

\begin{lstlisting}[language=C,frame=single]
  graph_t *graph_coalesce(const graph_t *original);
\end{lstlisting}

  Coalesce multiple edges in a pseudo-graph in a single one, returning a weighted true graph (i.e., \lstinline!graph_is_pseudo()! returns false).
  If \lstinline!original! is unweighted, the weight of the edge $E_{ij}$ is the number of edges between $V_i$ and $V_j$ in \lstinline!original!.
  If \lstinline!original! is weighted, the weight of the edge $E_{ij}$ is the sum of weights between $V_i$ and $V_j$ in \lstinline!original!.
  
  If \lstinline!original! isn't a pseudo-graph, or if memory was exhausted, the function returns \NULL.

\subsection{Input/Output}
\label{sec:graph-io}

\subsection{Insertion}
\label{sec:graph-insertion}

\subsection{Retrieval}
\label{sec:graph-retrieval}

\subsection{Removal}
\label{sec:graph-removal}

\subsection{Query}
\label{sec:graph-query}

\subsection{Adjacencies}
\label{sec:graph-adjacencies}

\subsection{Copying}
\label{sec:graph-copying}
