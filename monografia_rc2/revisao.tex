\chapter{Revis�o Bibliogr�fica}

\section{Considera��es Iniciais}

O estudo de Redes Complexas � a base para a compreens�o da complexidade, buscando explicar a emerg�ncia e 
evolu��o estrutural do esqueleto de um sistema complexo \cite{barabasi2007architecture}. O estudo da rede
muitas vezes ignora dados mais completos e se at�m apenas � presen�a de conex�es entre os n�s, o que permite
analisar a estrutura onde os processos ocorrem, fornecendo informa��es para explicar como comportamentos
emergem de partes mais simples.

\section{M�tricas}

M�tricas s�o uma maneira de sumarizar informa��es necess�rias para caracterizar a estrutura de uma rede.
A descri��o quantitativa das propriedades de redes fornecem ferramentas fundamentais para a an�lise de
redes reais e te�ricas, permitindo sua representa��o, caracteriza��o, classifica��o e modelagem 
\cite{costa2007characterization}.

\subsection{Distribui��o de grau}



\subsection{Distribui��o de \textit{clustering}}



\subsection{Assortatividade}



\subsection{K-Core}


\section{Modelos de rede}



\subsection{Modelo Erdos-Renyi}



\subsection{Modelo Barabasi-Albert}



\subsection{Modelo Watts-Strogatz}



\subsection{Modelo Ravasz-Barabasi}



