%%%%%%%%%%%%%%%%%%%%%%%%%%%%%%%%%%%%%%%%%%%%%%%%%%%%%%
% % Modelo de Monografia
% % Autor: Felipe Brunelli de Andrade - Mar�o 2010
%%%%%%%%%%%%%%%%%%%%%%%%%%%%%%%%%%%%%%%%%%%%%%%%%%%%%%

% % Defini��o da estrutura do documento, utilizando o ABNTex
% % Formato de impress�o frente-verso com come�o de cap�tulo em 
% % p�gina impar, com cabe�alho em cada p�gina com nome do cap�tulo
% % e n�mero de p�gina, A4 e fonte tamanho 12
\documentclass[espaco=duplo,brazil,ruledheader,a4paper,12pt,twoside]{abnt}
% % Pacotes para reconhecimento de acentos e hifeniza��o pt-br
\usepackage[brazil]{babel}
\usepackage[latin1]{inputenc}
% % \usepackage[dvips]{graphicx}
% % Pacote gr�fico para PDF
\usepackage[pdftex]{graphicx,lscape}
% % Estilo de refer�ncia para Bibliografia - Alfab�tico
%\usepackage{natbib}
\usepackage[pdftex,pdfpagelabels,plainpages=false,pagebackref,hyperfootnotes=false]{hyperref}
\hypersetup{colorlinks=true,linkcolor=blue,citecolor=blue,urlcolor=blue}
\usepackage[abnt-url-package=hyperref]{abnt-alf}
% % Pacote que prov� 11 simbolos matem�ticos
\usepackage{latexsym}
% % Pacote que permite a inclus�o de arquivos eps
\usepackage{psfrag}
% % pacote para adicionar endere�os eletr�nicos
\usepackage{url}
% % Pacote usado para listar Siglas e Simbolos
% \usepackage[estilo=UFPR]{tabela-simbolos}
\usepackage{tabela-simbolos}
% % Pacote semelhante ao minipage para inclus�o de figuras lado a lado
\usepackage{subfigure}
% % Pacote que auxilia em referencias - pouco usado
% \usepackage{varioref}
% % Suaviza��o de fonte
\usepackage{dsfont}
% % Pacote para adicionar espa�os quando necess�rio
% \usepackage{xspace}
% % Pacote que auxilia na cria��o de tabelas
% \usepackage{tabularx}
\usepackage{longtable}
\usepackage{multirow}
% % Pacote que ajuda a rotacionar elementos na p�gina
\usepackage{rotating}


% % Macros - use se precisar
% \newcommand{\simb}[2][]{
% \ifthenelse{\equal{#1}{}}
% {\addcontentsline{los}{simbolo}{\ensuremath{#2}}}
% {\addcontentsline{los}{simbolo}{#1}}
% \ensuremath{#2}}
% \makeatletter
% \newcommand{\listadesimbolos}{
% \pretextualchapter{Lista de S�mbolos}
% {\setlength{\parindent}{0cm}
% \@starttoc{los}}}
% \newcommand\l@simbolo[2]{\par #1, p.\thinspace#2}
% \makeatother
% 
% \newcommand{\R}{\mathds{R}}
% \newcommand{\Cinf}{\mathcal{C}^\infty}
% \newcommand{\Cinfc}{\Cinf_c}

\begin{document}

% % Tipos de imagens para serem inseridas no texto
\DeclareGraphicsExtensions{.jpg, .pdf, .mps, .png, .tiff} 

% % Defini��o de vari�veis para utiliza��o no ABNTex
\renewcommand{\autor}[0]{Bruno Kim Medeiros Cesar, Daniel Luiz de Albuquerque}
\renewcommand{\titulo}[0]{Dissemina��o de informa��o em Redes Complexas}
\renewcommand{\local}[0]{S�o Carlos -- SP}
\renewcommand{\instituicao}[0]{UNIVERSIDADE DE S�O PAULO}
\newcommand{\instituto}[0]{Instituto de Ci�ncias Matem�ticas e de Computa��o}
\renewcommand{\data}[0]{Junho de 2013}

\begin{titlepage}

\newcommand{\lyxline}[1]{
  {#1 \vspace{1ex} \hrule width \columnwidth \vspace{1ex}}
}

% Documentos a serem incluidos na Monografia
\thispagestyle{empty}

% capa - come�o
\begin{center}
\Huge{\textsf{\instituicao}}\\
%\small{\textsf{\instituto}}
\end{center}

\vspace{6cm}

\lyxline{}
\begin{center}
\noindent
\Large{
\textsf{\titulo}}\\
\vspace{1em}
\noindent
{\bf \autor}
\end{center}
\lyxline{}

\vspace{7cm}

\begin{center}
\textsf{\local}
\end{center}

\cleardoublepage
% folha de rosto - comeco
%\begin{center}
%Copyright 2010 \autor.\\ %Este documento � distribu�do nos termos da licen�a descrita no arquivo LICENCA que o acompanha.
%\end{center}
%\newpage

\vspace*{1cm}

\begin{center}
\noindent
\Large{
\textsf{\titulo}}
\end{center}

\vspace{2cm}

\begin{center}
\noindent
\large{\bf {\it \autor}}\\
\vspace{1em}
\end{center}

\vspace{3cm}

\hfill{
\begin{minipage}{0.6\columnwidth}
{Monografia referente ao segundo trabalho para a mat�ria Redes Complexas para a Computa��o.}\\\\\\\\\\

\end{minipage}}

\vspace{1.5cm}

\begin{center}
\textsf{USP - S�o Carlos}\\
\textsf{\data} \\
%\textsf{(Vers�o Final Revisada)}
\end{center}

%\newpage\ 
% folha de rosto - fim

\cleardoublepage
\begin{resumo}

\noindent{} Processos de dissemina��o em redes est�o presentes no estudo de propaga��o de epidemias, rumores, 
v�rus digitais e como mecanismo de atualiza��o de estado em sistemas distribu�dos. O sucesso de uma propaga��o, 
avaliado pelo n�mero de indiv�duos afetados e pelo tempo necess�rio para atingir o estado estacion�rio,
depende tanto de caracter�sticas estruturais da rede como de par�metros do modelo de propaga��o. LACUNA. 
Este trabalho simula diversos modelos s�ncronos de dissemina��o de informa��o em redes reais e sint�ticas, 
visando identificar caracter�sticas estruturais que afetem o sucesso da propaga��o. 

\end{resumo}


\end{titlepage}
% \tableofcontents
\sumario
\listoffigures
\listoftables

\begin{center}
\chapter*{Lista de Abreviaturas}
\end{center}

\singlespacing

\noindent
 
\begin{tabular}{p{2.5cm} p{12cm}}
RC & \textit{Redes Complexas}\\
\end{tabular}


\clearpage

\cleardoublepage

%\doublespacing

\pagenumbering{arabic}

\chapter{Introdu��o}

Processos de dissemina��o de informa��o s�o encontrados em muitas �reas do conhecimento, como na dispers�o de rumores,
propaga��o de v�rus digitais e biol�gicos, atualiza��o de estado em sistemas distribu�dos, e mesmo em processos 
f�sicos de dispers�o. O elemento comum entre todos estes fen�menos � a ocorr�ncia de uma atividade descentralizada,
onde cada agente interage com seus pares para alterar o estado global do sistema. A dissemina��o ent�o emerge
como um comportamento complexo com origem na intera��o entre agentes simples.

No primeiro trabalho da disciplina, examinamos diferentes modelos de dissemina��o de informa��o, 
provenientes das �reas de epidemiologia e sociologia. Tais modelos consistem na identifica��o de compartimentos
que correspondem a diferentes est�gios de uma epidemia, e o estabelecimento de regras para transi��o entre 
compartimentos, dependente ou n�o da intera��o entre indiv�duos. 

Tal abordagem prov� uma equa��o diferencial que pode ser solucionada por m�todos de integra��o num�rica, correspondendo
ao modelo idealizado de uma popula��o muito grande onde os indiv�duos se encontram aleatoriamente ao acaso. Tal abordagem
cont�nua permite um estudo anal�tico profundo das condi��es e resultados da dissemina��o, mas se baseia em um modelo de
intera��o - mistura completa - que n�o tem sentido em muitos processos reais.

O modelo de mistura completa assume que os indiv�duos possuem igual capacidade para interagir com quaisquer outros 
indiv�duos. � poss�vel implementar estratifica��es \textit{ad-hoc} adicionando mais compartimentos ao sistema, mas uma
abordagem mais promissora � simplesmente implementar o processo sobre uma rede que mapeia todas as rela��es poss�veis.
Deste modo, o modelo de mistura completa torna-se um caso especial, onde a rede � um grafo completo. Deste modo,
abandonam-se an�lises cont�nuas em favor da simula��o discreta, onde a rede oferece a topologia das intera��es.

Contudo, com isso adiciona-se uma nova camada de complexidade � an�lise, pois os efeitos da dissemina��o tamb�m dependem da 
estrutura da rede sobre a qual ela � executada. A pesquisa sobre esse tipo de processo din�mico ainda � 

Linkar trabalho anterior, Impacto dos modelos ded rede na propaga��o; M�tricas estruturais da rede como medida comparativa.  trabalho analisa propaga��o em diferentes aspectos: modelos de redes e redes reais;

\chapter{Revis�o Bibliogr�fica}

O estudo de Redes Complexas � a base para a compreens�o da complexidade, buscando explicar a emerg�ncia e 
evolu��o estrutural do esqueleto de um sistema complexo \cite{barabasi2007architecture}. O estudo da rede
muitas vezes ignora dados mais completos e se at�m apenas � presen�a de conex�es entre os n�s, o que permite
analisar a estrutura onde os processos ocorrem, fornecendo informa��es para explicar como comportamentos
emergem de partes mais simples.

\section{M�tricas}

M�tricas s�o uma maneira de sumarizar informa��es necess�rias para caracterizar a estrutura de uma rede.
A descri��o quantitativa das propriedades de redes fornecem ferramentas fundamentais para a an�lise de
redes reais e te�ricas, permitindo sua representa��o, caracteriza��o, classifica��o e modelagem 
\cite{costa2007characterization}.

Nas m�tricas a seguir um grafo com $N$ v�rtices � representado por uma matriz $A \in \mathbb{R}^{N \times N}$, 
onde $A_{ij}$ � 1 caso exista uma aresta entre os v�rtices $V_i$ e $V_j$, e 0 caso contr�rio. Grafos n�o-direcionados 
necessariamente s�o representados por uma matriz sim�trica. Sup�e-se que o grafo n�o possui auto-adjac�ncia, isto �, 
$A_{ii} = 0, \forall i$. O n�mero de arestas $M$ pode ser calculado por $\frac{1}{2} \sum A_{ij}$ para grafos 
n�o-direcionados e por $\sum A_{ij}$ para grafos direcionados. 

Um ponto de interesse � a an�lise de \emph{distribui��es de m�tricas}, onde a cada v�rtice � associada uma 
medida, e analisamos o histograma destas medidas como amostras de uma distribui��o aleat�ria. Tal distribui��o
oferece informa��o sobre o comportamento global da rede, e revela o processo de forma��o da rede.

\subsection{Distribui��o de grau}

O \emph{grau} $k_i$ de um v�rtice $V_i$ � o n�mero de arestas que est�o conectadas a tal v�rtice. 
Em grafos n�o-direcionados calcula-se o grau pela Equa��o \ref{eq:deg-undirected}. Em grafos direcionados pode-se
tamb�m medir  
o n�mero de arestas de entrade e de sa�da, representadas respectivamente por $k^{in}_i$ e $k^{out}_i$, 
dadas pela Equa��o \ref{eq:deg-directed}. 

\begin{alignat}{4}
 k_i &= \sum_{j} A_{ij}      &\quad          &                  &\quad           &                  &\quad \text{Grafos n�o-direcionados}\label{eq:deg-undirected} \\
 k_i &= k_i^{out} + k_i^{in} &\quad k_i^{in} &= \sum_{j} A_{ji} &\quad k_i^{out} &= \sum_{j} A_{ij} &\quad \text{Grafos direcionados}\label{eq:deg-directed}
\end{alignat}

O \emph{grau m�dio} $\langle k \rangle$ de uma rede � a m�dia de graus para todos os v�rtices, como calculado
na Equa��o \ref{eq:deg-average}. Tal medida � fundamental para muitos modelos.

\begin{equation}
 \langle k \rangle = \frac{1}{N} \sum_{i} k_i = \frac{M}{N}
 \label{eq:deg-average}
\end{equation}

A an�lise da \emph{distribui��o de graus} � uma informa��o importante sobre uma rede, fornecendo detalhes
sobre sua estrutura e o comportamento de processos sobre ela. Em espec�fico, no processo de dissemina��o de informa��es,
a exist�ncia de um limiar de infec��o depende da presen�a do segundo momento da distribui��o; 
para redes com vari�ncia infinita, este limiar � zero, indicando que a dissemina��o n�o pode ser contida uma vez iniciada
em tais redes.

Uma distribui��o \emph{exponencial} implica que o grau m�dio $\langle k \rangle$ � o grau t�pico da rede, com a 
ocorr�ncia de v�rtices com grau muito maior ou muito menor exponencialmente raras. Redes com distribui��o de grau 
do tipo Poisson (Equa��o \ref{eq:deg-poisson}) s�o consideradas redes exponenciais.

Uma distribui��o \emph{livre de escala} implica que n�o existe um grau t�pico da rede. Esta propriedade � verificada com a
ocorr�ncia de poucos v�rtices com grau muito acima da m�dia. O principal tipo de distribui��o � \emph{de Pareto} ou 
\emph{Lei de Pot�ncia}, dado pela Equa��o \ref{eq:deg-pareto}, e redes reais possuem coeficiente $\gamma$ entre 2 e 3 
\cite{costa2007characterization}. 
Uma propriedade interessante deste tipo de distribui��o � que o n�mero de momentos definidos $m$ obedece 
$m < \gamma-1$. Deste modo, com $\gamma < 3$, a vari�ncia � infinita e apenas a m�dia pode ser definida. 
Observe-se que isto � v�lido para a distribui��o te�rica; os graus da rede s�o amostras de uma tal distribui��o, 
e sendo em n�mero finito, permitem calcular at� $N$ momentos.

\begin{align}
 P(k; \langle k \rangle) &\sim \frac{{\langle k \rangle}^k e^{-\langle k \rangle}}{k!} \label{eq:deg-poisson} \\
 P(k; \gamma) &\sim k^{-\gamma} \label{eq:deg-pareto}
\end{align}

\subsection{Distribui��o de \textit{clustering}}

Redes reais, especialmente redes sociais, possuem um n�mero elevado de la�os pequenos e subconjuntos de v�rtices 
altamente interconectados. Esta propriedade � chamada de \textit{clustering} (do ingl�s, ``agrupamento''),
propriedade que est� relacionada (mas n�o limitada) � ocorr�ncia de comunidades. As m�tricas que ser�o apresentadas
nos par�grafos seguintes s�o definidas apenas para grafos n�o-direcionados.

A \emph{transitividade} � uma medida da densidade de tri�ngulos (isto �, la�os de comprimento 3)
na rede. Ela � definida na Equa��o \ref{eq:transitivity}, onde $N_3$ � o n�mero de triplas, enquanto $N_\bigtriangleup$
� o n�mero de triplas fechadas. 

\begin{align}
 C &= \frac{3 N_\bigtriangleup}{N_3} \label{eq:transitivity} \\
 N_\bigtriangleup &= \sum_{k > j > i} A_{ij} A_{jk} A_{ki} \nonumber \\
 N_3 &= \sum_{k > j > i} (A_{ij} A_{ik} + A_{ji} A_{jk} + A_{ki} A_{kj}) \nonumber
\end{align}

O \emph{coeficiente de agrupamento local} $C_i$ mede o \textit{clustering} para cada v�rtice,
considerando apenas informa��o dos v�rtices pr�ximos. Ele � calculado pela Equa��o \ref{eq:local-clustering}, onde $l_i$
� o n�mero de arestas entre os vizinhos de $V_i$, e $k_i (k_i - 1)$ � o n�mero de arestas
poss�veis entre os vizinhos de $V_i$. Em outras palavras, esta medida fornece o qu�o completo � o subgrafo 
que cont�m os vizinhos de $V_i$.

\begin{align}
 C_i &= \frac{2 l_i}{k_i (k_i - 1)} \label{eq:local-clustering} \\
 l_i &= \sum_{k > j} A_{ij} A_{jk} A_{ki} \nonumber
\end{align}

A m�dia dos coeficientes de agrupamento locais $\langle C \rangle$ � um resumo da distribui��o completa, 
e n�o deve ser confundida com a transitividade $C$: a primeira d� peso igual para cada v�rtice, 
enquanto a segunda d� peso igual para cada tri�ngulo.

\subsection{Assortatividade}

Redes de mundo real frequentemente apresentam tipos diferentes de v�rtices, e a probabilidade de conex�o 
entre v�rtices depende destes tipos \cite{newman2003structure}. Quando n�s em uma rede tendem a se
conectar com n�s semelhantes a rede � dita \emph{assortativa}; do contr�rio, � dita \emph{desassortativa}. 

Um caso especial de an�lise de assortatividade � definir o tipo de um v�rtice com base no seu 
grau. Seja $k_{nn}(i)$ a m�dia do grau dos vizinhos (\textit{nearest neighbors}) do v�rtice $V_i$, 
calculado pela Equa��o \ref{eq:deg-avg-degree}. A assortatividade $r$ de uma rede � definida como o 
coeficiente de Pearson da distribui��o das vari�veis $k$ e $k_{nn}$ \cite{newman2002assortative}, 
com valor positivo para redes assortativas e negativo para redes desassortativas. 
A Equa��o \ref{eq:pearson} mostra um procedimento de c�lculo de $r$, onde $\langle \cdot \rangle$ 
denota a m�dia e $\sigma(\cdot)$ denota o desvio padr�o da distribui��o.

\begin{align}
 k_{nn}(i) &= \frac{1}{k_i} \sum_j A_{ij} k_j \label{eq:deg-avg-degree} \\
 r        &= \sum_i \left( \frac{k_i - \langle k \rangle}{\sigma(k)} \right) 
                    \left( \frac{k_{nn}(i) - \langle k_{nn} \rangle}{\sigma(k_{nn})} \right)
             \label{eq:pearson}
\end{align}

\subsection{\textit{$K$-Core}}

Algumas redes tendem a se organizar em camadas, onde o n�cleo consiste em um conjunto altamente interconectado
de v�rtices e camadas sucessivas apresentam menor interconex�o. A m�trica \textit{$k$-core} determina estas 
camadas definindo como n�cleo $k$ o maior subgrafo conectado que cont�m apenas v�rtices com grau $k$
ou maior. Um v�rtice no n�cleo $k$ tamb�m est� inclu�do no n�cleo $k-1$, e seu valor de \textit{coreness}
� definido como o maior n�cleo ao qual ele pertence.

� poss�vel determinar um n�cleo $k$ com um algoritmo $\mathcal{O}(N)$, removendo repetidamente todo v�rtice
com grau menor que $k$. Todos os v�rtices remanescentes possuem \textit{coreness} $k$. Dissemina��es que se 
iniciam no n�cleo central da rede - isto �, em algum v�rtice com maior \textit{coreness} - afetam um n�mero 
maior de v�rtices do que se iniciarem em n�s com grau equivalente, mas em camadas mais perif�ricas 
\cite{stanley2010identification}.

\section{Modelos de rede}
\label{sec:modelos}

\subsection{Modelo Erd�s-R�nyi}

O modelo de redes proposto por Erdos e R�nyi \cite{erd6s1960evolution} consiste no modelo de redes aleat�rias. A cria��o de uma rede aleat�ria se inicia com $N$ v�rtices desconectados. A cada itera��o escolhe-se um par de v�rtices aleatoriamente para serem conectados. Nota-se que quanto maior o n�mero de itera��es $M$, maiores as chances de selecionar v�rtices que j� possuam uma conex�o, portanto, mais densamente conectado se tornam determinadas �reas de uma rede, evidenciando a gera��o de \textit{clusters}. Em suma, s�o adicionadas $M$ arestas de forma aleat�ria em um n�mero fixo de $N$ v�rtices.

Erd�s e R�nyi estudaram as mudan�as na topologia de redes aleat�rias em fun��o de $M$. Eles constataram que quanto menor o $M$, a rede tende a se fragmentar em muitos \textit{clusters} pequenos, os quais s�o chamados de componentes. � medida que $M$ aumenta, os componentes tamb�m aumentam, primeiramente em fun��o de intraconex�es entre v�rtices isolados de um componente e posteriormente em fun��o de interconex�es entre os componentes, como � apresentado na Figura \ref{graph_random}.

\begin{figure}[!htb]
\centering
\includegraphics[scale=0.8]{./imagens/graph_random.png}
\caption{Exemplo de rede aleat�ria \cite{strogatz2001exploring}}
\label{graph_random}
\end{figure}

Desde o trabalho pioneiro, redes aleat�rias t�m sido estudadas mais profundamente sob o aspecto matem�tico\cite{bollobas2001random}. A partir deste modelo foram criadas arquiteturas idealizadas para modelos din�micos de redes de genes, ecossistemas e a propaga��o de doen�as infecciosas \cite{strogatz2001exploring}. Outros modelos foram propostos devido � limita��o deste modelo em representar caracter�sticas de redes reais.

\subsection{Modelo Watts-Strogatz}

Redes regulares e aleat�rias s�o ambas boas idealiza��es e t�m suas aplica��es no campo da pesquisa \cite{strogatz2001exploring}, por�m, a maioria das redes reais t�m um padr�o intermedi�rio entre a ordem e a desordem. O modelo proposto por Watts e Strogatz \cite{watts1998collective}, denominado \textit{small world}, � capaz de ajustar este meio-termo a partir de uma rede regular. O nome \textit{small world} (``mundo pequeno'', em ingl�s), faz refer�ncia ao trabalho de Stanley Milgram, no qual constatou-se que, sob o ponto de vista matem�tico, quaisquer duas pessoas nos Estados Unidos est�o separadas por 5 conhecidos \cite{milgram1967small}. John Guare popularizou o termo \textit{small world} como ``Seis Graus de Separa��o'' em seu livro, considerando a premissa existencial de que quaisquer duas pessoas no mundo est�o conectadas por n�o mais do que 6 conhecidos \cite{guare1992six}.

Na interpola��o entre redes regulares e aleat�rias, � considerado o processo de religa��o aleat�ria apresentado na Figura \ref{fig:graph_world}. Considerando uma rede com a estrutura em anel contendo $N$ v�rtices e $k$ arestas por v�rtice, percorre-se todas as arestas da rede, religando-as de forma aleat�ria segundo uma probabilidade $p$. Se $p = 0$ a aresta permanecer� intacta, pois n�o existe probabilidade alguma de religa��o; no extremo oposto, com $p = 1$, a aresta ser� religada de forma aleat�ria na rede. A escolha do valor de $p$ entre $0 < p < 1$ permite variar entre uma rede aleat�ria e uma rede regular.

\begin{figure}[!htb]
\centering
\includegraphics[scale=0.8]{./imagens/graph_world.png}
\caption{Processo de obten��o de uma rede de pequeno-mundo a partir de uma rede regular}
\label{fig:graph_world}
\end{figure}
 
As redes \textit{small world} apresentam duas propriedades importantes \cite{strogatz2001exploring}: (1) Redes altamente agrupadas - Estas redes s�o muito mais agrupadas do que redes aleat�rias. Considerando os v�rtices A, B e C, se A est� ligado a B e B est� ligado a C, ent�o existe grande probabilidade de A estar conectado a C; e (2) Caminho geod�sico m�dio baixo - A maiores dos v�rtices est�o conectados atrav�s de um caminho geod�sico de comprimento baixo. O comportamento com rela��o aos caminhos geod�sicos pode ser constatado atrav�s da Equa��o \ref{eq:caminho}, que consiste no c�lculo do caminho m�nimo m�dio entre todos os pares de v�rtices em um grafo n�o-direcionado, onde $d_{ij}$ � a dist�ncia geod�sica entre o v�rtice $i$ e o v�rtice $j$

\begin{equation}
  L = \frac{1}{\frac{1}{2} n(n+1)} \sum_{i \geq j}d_{ij}\\
  \label{eq:caminho}
\end{equation}

Matematicamente, estas propriedades podem ser evidenciadas com c�lculos estat�sticos. Primeiramente vale ressaltar que para que as propriedades sejam v�lidas assume-se que o grafo n�o possui v�rtices desconexos, necessitando que $n > k > \log (n) > 1$, no qual $k > \log (n)$ garante que um grafo aleat�rio seja conectado. 

Watts e Strogatz constataram que estas duas propriedades est�o presentes tamb�m em muitas redes naturais e tecnol�gicas \cite{watts1998collective}, como a rede neural do verme \textit{Caenorhabditis Elegans}, a rede de energia do oeste do Estados Unidos e o grafo de colabora��o entre autores em filmes. Al�m disso, constatou-se que sistemas din�micos com efeito \textit{small world} apresentam aumento na velocidade de propaga��o do sinal, poder computacional e sincroniza��o. Em geral, os caminhos m�nimos prov�m canais de comunica��o de alta velocidade entre diferentes partes do sistema, facilitando os processos din�micos, como sincroniza��o, que requer informa��o com rela��o ao estado global do sistema \cite{strogatz2001exploring}. Al�m disso, doen�as infecciosas se espalham mais rapidamente em redes de \textit{small world}.

\subsection{Modelo Barab�si-Albert}

Em redes reais, alguns n�s s�o mais conectados do que outros. No modelo proposto por Barab�si e Albert \cite{barabasi1999emergence}, denominado Livre de Escala, as redes apresentam uma ordem na din�mica de estrutura��o. As redes Livres de Escala possuem duas caracter�sticas chave presentes em redes reais: (1) Incorporam crescimento; e (2) Apresentam conex�o preferencial. A conex�o preferencial se refere � tend�ncia de um novo v�rtice conectar-se a um v�rtice de grau elevado. Estes n�s altamente conectados s�o denominados \textit{hubs}.

Barab�si e Albert afirmam que em muitos sistemas a probabilidade $P(k)$ de que um v�rtice interaja com $k$ v�rtices decai como uma Lei de Pot�ncia, sendo $P(k) \sim k^{-\gamma}$, ou seja, decai muito mais lentamente que Poisson, distribui��o de graus prevista em uma rede aleat�ria ou \textit{small world} \cite{strogatz2001exploring}. Na Lei de Pot�ncia, o expoente $\gamma$ determina a taxa de decaimento, que � tipicamente medida no intervalo $2 < \gamma < 3$. Na rede da Web, a probabilidade de $k$ documentos apontarem para uma certa p�gina segue a Lei de Pot�ncia com $\gamma = 2.1 \pm 0.1$ \cite{barabasi1995emergence}. 

Em outras palavras, uma rede aleat�ria possui grande parte dos v�rtices com n�mero de conex�es pr�ximo da m�dia da distribui��o, de forma que cada aresta est� presente ou ausente com probabilidade igual. Diferentemente, em redes livres de escala a distribui��o do grau dos v�rtices decai lentamente para a direita, formando uma "cauda", indicando que a maioria dos n�s tem grau menor que a m�dia da distribui��o \cite{watts2004new}, de forma que a probabilidade da exist�ncia de um v�rtice com grau alto � muito menor do que um v�rtice com grau baixo. Este comportamento � ilustrado na Figura \ref{comp_distr}.

\begin{figure}[!htb]
\centering
\includegraphics[scale=0.15]{./imagens/comparisson.png}
\caption{Compara��o entre a distribui��o de uma rede aleat�ria e uma rede livre de escala \cite{scott2011network}}
\label{comp_distr}
\end{figure}

Este comportamento � notado em grande parte das redes reais. No Twitter, pessoas normalmente seguem amigos e pessoas famosas que disp�em informa��es de sua vida pessoal. Em uma rede de cita��es de artigos, � evidente que pesquisadores t�m prefer�ncia em citar trabalhos conhecidos e bem publicados. Um exemplo de uma rede Livre de Escala � apresentado na Figura \ref{graph_scale}.

\begin{figure}[!htb]
\centering
\includegraphics[scale=0.8]{./imagens/scale_free.png}
\caption{Exemplo de rede Livre de Escala \cite{strogatz2001exploring}}
\label{graph_scale}
\end{figure}

\subsection{Modelo Ravasz-Barab�si}

Muitas redes reais que modelam fen�menos naturais e sociais possuem duas propriedades gen�ricas: (1) s�o livres de escala e (2) apresentam um alto grau de agrupamento. Resultados emp�ricos mostraram que o coeficiente de agrupamento m�dio � significamente maior para redes reais do que para redes aleat�rias de mesmo tamanho. Ao mesmo tempo muitas redes cient�ficas e biol�gicas demonstraram ter a propriedade de serem livres de escala. Modelos propostos para descrever a topologia de redes complexas, em geral, t�m dificuldade em apresentar tais caracter�sticas simultaneamente.

Ravasz e Barab�si \cite{ravasz2003hierarchical} mostraram que estas duas propriedades ocorrem devido a presen�a de uma organiza��o hier�rquica. Portanto, o modelo proposto por ambos consiste em uma rede que combina a propriedade de redes Livre de Escala e de alto grau de agrupamento.

\begin{figure}[!htb]
\centering
\includegraphics[scale=0.4]{./imagens/graph_h.png}
\caption{Exemplo do processo de forma��o de uma rede hier�rquica \cite{ravasz2003hierarchical}} 
\label{graph_h}
\end{figure}

\chapter{Desenvolvimento do Trabalho}

\section{Considera��es Iniciais}



\section{Atividades Realizadas}

\subsection{Biblioteca Cgraph}

 A biblioteca CGraph \cite{cesar2013cgraph} utilizada foi implementada por um dos autores (Cesar) 
durante seu trabalho de Inicia��o Cient�fica e Mestrado, utilizando a linguagem de programa��o C89 ANSI 
em ambiente POSIX. A biblioteca foi extendida por ambos os autores para an�lise de processos de dissemina��o.

A estrutura de dados \texttt{graph\_t} representa um grafo por meio de lista de adjac�ncias, armazenando 
toda a informa��o necess�ria em mem�ria f�sica. Para um sistema com 4 GB de mem�ria RAM e arquitetura de 
64 bits, um grafo direcionado est� limitado a menos de 500 milh�es de arestas, e um grafo n�o-direcionado 
est� limitado a menos de 250 milh�es de arestas.

A biblioteca ainda est� incompleta, mas j� possui diversos algoritmos para entrada, gera��o, m�trica e 
simula��o de grafos n�o-direcionados, apresentados a seguir:

\begin{itemize}
 \item Leitura de \textit{datasets}.
 \item Inser��o de arestas, extra��o de subgrafo.
 \item Extra��o de componentes.
 \item M�tricas de centralidade: grau, intermedia��o, proximidade, autovetor, PageRank e $k$-core.
 \item C�lculo de dist�ncia geod�sica.
 \item Assortatividade.
 \item Modelos de rede: grafo completo, Erd�s-R�nyi, Barab�si-Albert, Watts-Strogatz, Ravasz-Barab�si.
 \item Layout com sa�da SVG: aleat�rio, circular, circular ordenado por grau, definido pelo usu�rio.
 \item Modelos de dissemina��o: SI, SIS, SIR, SEIR, Daley-Kendall, zumbi.
\end{itemize}

A simula��o de uma propaga��o � realizada de forma abstra�da: a fun��o \texttt{graph\_propagation\_t}
executa os passos, contando o n�mero de indiv�duos infecciosos e escolhendo uniformemente entre seus
adjacentes quem recebe uma \emph{mensagem}. O modelo de propaga��o, representado na estrutura 
\texttt{graph\_propagation\_model\_t}, � respons�vel por implementar dois \textit{callbacks} de
forma semelhante � heran�a em linguagens que permitem orienta��o a objeto:

\begin{description}
 \item[\texttt{graph\_transition\_t}] 
  Determina a transi��o dos indiv�duos, dependendo do estado atual e das mensagens transmitidas.
 \item[\texttt{graph\_is\_end\_t}]
  Determina se a propaga��o j� atingiu um estado estacion�rio.
\end{description}

Esta implementa��o de propaga��o n�o admite modelos onde mais de um indiv�duo pode ser infectado a cada passo,
como o modelo em cascata de Borge et al \cite{borge2011structural}, ou modelos s�ncronos, onde apenas um 
infeccioso dissemina por turno.

\subsection{Propaga��o em datasets e em modelos equivalentes}



\subsection{Compara��o em solu��o num�rica e solu��o em redes complexas}



\subsection{Compara��o entre modelos de propaga��o com $R_0$ equivalente}



\subsection{Aplica��o de propaga��o de informa��o em poblemas reais}



\subsubsection{Gripe su�na em Hong Kong}



\subsubsection{Sobreviv�ncia ao Apocalipse Zombie}



\section{Resultados Obtidos}



\chapter{Conclus�o}

\section{Contribui��es}

Este trabalho apresentou os resultados da simula��o de dissemina��o de informa��o sobre redes complexas, 
apresentando os impactos que diferentes estruturas de redes implicam no sucesso de uma propaga��o. Para 
isto foi implementada e disponibilizada uma biblioteca de c�digo aberto com capacidade para simula��o de 
dissemina��o s�ncrona, al�m do tratamento e an�lise de v�rios \textit{datasets} dispon�veis em diferentes 
bases de dados.

\section{Trabalhos Futuros}

Este trabalho apenas come�ou a explorar as possibilidades de propaga��o de informa��o. A biblioteca ainda
deve ser extendida para permitir propaga��es s�ncronas e com m�ltiplos contatos, para avaliar como a 
estrutura da rede altera tais resultados. Tamb�m devem ser avaliados os efeitos de diferentes pontos 
iniciais de propaga��o, e a sele��o de tais pontos a partir de sua localiza��o relativa na rede. Ainda,
deve-se avaliar o n�mero total de intera��es, uma outra importante m�trica de sucesso desconsiderada neste
trabalho.




% % Usado se voc� quiser utilizar mais de uma refer�ncia que n�o foi citado no texto
% \nocite{*}

\bibliographystyle{abnt-alf}
\bibliography{referencias}

\apendice
\include{apendice}

\end{document}