\documentclass[a4paper,10pt]{article}
\usepackage[utf8]{inputenc}
\usepackage{amsmath}
% Title Page
\title{CGraph documentation}
\author{Bruno Kim Medeiros Cesar}


\begin{document}
\maketitle


\begin{abstract}
\end{abstract}

\section{\texttt{sorting}}
\section{\texttt{list}}
\section{\texttt{set}}
\section{\texttt{graph}}
\section{\texttt{graph\_metric}}

\subsection{Constants}

These constants are hard-coded to protect some numeric processes of hanging.
They can be redefined during compilation, passing a flag such as 

\texttt{-DGRAPH\_METRIC\_TOLERANCE=1E-3}.

\subsubsection{\texttt{GRAPH\_METRIC\_TOLERANCE}}

Error tolerance for numeric methods.

\subsubsection{\texttt{GRAPH\_METRIC\_MAX\_ITERATIONS}}

Maximum number of iterations for numeric methods.

\subsection{Component identification and extraction}

\subsubsection{\texttt{graph\_undirected\_components}}

Label vertices' components treating edges as undirected.

For directed graphs, considers adjacencies as incidences.
Labels start from 0 and are sequential with step 1.
Component IDs are not ordered according to size.

\subsubsection{\texttt{graph\_directed\_components}}

Label vertices' components treating edges as directed.

For undirected graphs, simply call \texttt{graph\_undirected\_components}. For 
directed graphs, two vertices $v_i$ and $v_j$ are in the same component if and
only if
\begin{align*}
 &d(v_i, v_j) \neq \infty \\
 &d(v_j, v_i) \neq \infty
\end{align*}

where $d(u,v)$ is the geodesic distance between them. In other words, they are
in the same component if they are mutually reachable.

Labels start from 0 and are sequential with step 1.
Component IDs are not ordered according to size.

\end{document}
