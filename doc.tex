\documentclass[a4paper,10pt]{article}
\usepackage[utf8]{inputenc}
\usepackage{amsmath}
\usepackage{hyperref}

\title{CGraph documentation}
\author{Bruno Kim Medeiros Cesar}

\begin{document}
\maketitle

\begin{abstract}
\end{abstract}

\tableofcontents

\section{\texttt{sorting}}
\section{\texttt{list}}
\section{\texttt{set}}
\section{\texttt{graph}}
\section{\texttt{graph\_metric}}

\subsection{Constants}

These constants are hard-coded to protect some numeric processes of hanging.
They can be redefined during compilation, passing a flag such as 

\texttt{-DGRAPH\_METRIC\_TOLERANCE=1E-3}.

\subsubsection{\texttt{GRAPH\_METRIC\_TOLERANCE}}

Error tolerance for numeric methods.

\subsubsection{\texttt{GRAPH\_METRIC\_MAX\_ITERATIONS}}

Maximum number of iterations for numeric methods.

\subsection{Component identification and extraction}

\subsubsection{\texttt{graph\_undirected\_components}}

Label vertices' components treating edges as undirected.

\begin{description}
 \item[Preconditions] \texttt{label} must have dimension $n$.
 \item[Postconditions] \texttt{label[i]} is the component ID of vertex $v_i$.
 \item[Return] Number of components
\end{description}

For directed graphs, considers adjacencies as incidences.
Labels start from 0 and are sequential with step 1.
Component IDs are not ordered according to size.

\subsubsection{\texttt{graph\_directed\_components}}

Label vertices' components treating edges as directed. NOT IMPLEMENTED YET.

\begin{description}
 \item[Preconditions] \texttt{label} must have dimension $n$.
 \item[Postconditions] \texttt{label[i]} is the component ID of vertex $v_i$.
 \item[Return] Number of components
\end{description}

For undirected graphs, simply call \texttt{graph\_undirected\_components}. For 
directed graphs, two vertices $v_i$ and $v_j$ are in the same component if and
only if
\begin{align*}
 &d(v_i, v_j) \neq \infty \\
 &d(v_j, v_i) \neq \infty
\end{align*}

where $d(u,v)$ is the geodesic distance between them. In other words, they are
in the same component if they are mutually reachable.

Labels start from 0 and are sequential with step 1.
Component IDs are not ordered according to size.

\subsubsection{\texttt{graph\_num\_components}}

Extract number of components from label vector.

\begin{description}
 \item[Preconditions] ~\\
   $n > 0$ \\
   \texttt{label} must have dimension $n$. \\
   \texttt{label} must contain sequential IDs starting from 0.
 \item[Return] Number of components
\end{description}

\subsubsection{\texttt{graph\_components}}

Map components to vertices from label vector.

\begin{description}
 \item[Preconditions] ~\\
   $n > 0$ \\
   \texttt{label} must have dimension $n$. \\
   \texttt{label} must contain sequential IDs starting from 0. \\
   \texttt{comp} must have size \texttt{num\_comp} and all sets should be already
   initialized. \\
   \texttt{graph\_num\_components(g) == num\_comp}
 \item[Postconditions] ~\\
   If $v_i$ is in component $c_j$, then \\
    \quad\texttt{label[i] == j} and \\
    \quad\texttt{set\_contains(comp[j], i)} is true.
 \item[Return] Number of components
\end{description}

\subsubsection{\texttt{graph\_components}}

Creates a new graph from \texttt{g}'s largest component.

The guarantee of vertices' order ID is the same as \texttt{graph\_subset}.
If two or more components have the same maximum size, one will be chosen in
an undefined way.

\begin{description}
 \item[Return] A new graph isomorphic to \texttt{g}'s largest component.
 \item[Memory deallocation] ~\\
   \texttt{graph\_t *largest = graph\_components(g);} \\
   \texttt{delete\_graph(largest);}
\end{description}

\subsection{Degree metrics}
\subsubsection{\texttt{graph\_degree}}

List all vertices' degrees.

\begin{description}
 \item[Preconditions] \texttt{degree} must have dimension $n$.
 \item[Postconditions] \texttt{degree[i]} is the degree of vertex $v_i$.
\end{description}

The degree of a directed graph's vertex is defined as the sum of incoming
and outgoing edges.

\subsubsection{\texttt{graph\_directed\_degree}}

List all vertices' incoming and outgoing degrees.

\begin{description}
 \item[Preconditions] ~\\
   \texttt{g} must be directed.
   \texttt{in\_degree} must have dimension $n$.
   \texttt{out\_degree} must have dimension $n$.
 \item[Postconditions] ~\\
   \texttt{in\_degree[i]} is the number of incoming edges to vertex $v_i$.
   \texttt{out\_degree[i]} is the number of outgoing edges from vertex $v_i$.
\end{description}

\subsection{Clustering metrics}
\subsubsection{\texttt{graph\_clustering}}
List all vertices' local clustering.

\begin{description}
 \item[Preconditions] ~\\
   \texttt{g} must be undirected. \\
   \texttt{clustering} must have dimension $n$.
 \item[Postconditions] \texttt{clustering[i]} is the local clustering coefficient
  of vertex $v_i$.
\end{description}


The local clustering coefficient is only defined for undirected graphs, and
gives the ratio of edges between a vertex' neighbors and all possible edges.

Formally,

\begin{equation*}
 C_i = \frac{e_i}{\binom{k_i}{2}} = \frac{2 e_i}{k_i (k_i - 1)}
\end{equation*}

where

\begin{description}
 \item[$C_i$] is the local clustering coefficient of vertex $v_i$.
 \item[$e_i$] is the number of edges between $v_i$'s neighbors.
 \item[$k_i$] is the degree of $v_i$.
\end{description}

If a vertex $v_i$ has 0 or 1 adjacents, $C_i = 0$ by definition.
 
\subsubsection{\texttt{graph\_num\_triplets}}
Counts number of triplets and triangles (6 * number of closed triplets).
\subsubsection{\texttt{graph\_transitivity}}
Compute the ratio between number of triangles and number of triplets.

\subsection{Geodesic distance metrics}
\subsubsection{Definitions}
\subsubsection{\texttt{graph\_geodesic\_distance}}
\subsubsection{\texttt{graph\_geodesic\_vertex}}
\subsubsection{\texttt{graph\_geodesic\_all}}
\subsubsection{\texttt{graph\_geodesic\_distribution}}

\subsection{Centrality measures}
\subsubsection{\texttt{graph\_betweenness}}
\subsubsection{\texttt{graph\_eigenvector}}
\subsubsection{\texttt{graph\_pagerank}}
\subsubsection{\texttt{graph\_kcore}}

\subsection{Correlation measures}
\subsubsection{\texttt{graph\_degree\_matrix}}
\subsubsection{\texttt{graph\_neighbor\_degree\_vertex}}
\subsubsection{\texttt{graph\_neighbor\_degree\_all}}
\subsubsection{\texttt{graph\_knn}}
\subsubsection{\texttt{graph\_assortativity}}

\section{\texttt{graph\_layout}}

\subsection{Types}

\subsubsection{\texttt{coord\_t}}

Euclidean coordinates in 2D.

\subsubsection{\texttt{box\_t}}

Box (rectangle) definition in 2D, given by its SW and NE vertices in a 
positively oriented world frame, such as the screen. Images may have
a negatively oriented frame, with $y$ pointing down.
It is necessary that \texttt{box.sw.y < box.ne.y} and \texttt{box.sw.x < box.ne.x}.

\subsubsection{\texttt{circle\_style\_t}}

SVG circle style.

\begin{description}
 \item[\texttt{radius}] Circle radius in pixels.
 \item[\texttt{width}] Stroke width in pixels. This is added to the radius for total
 size.
 \item[\texttt{color}] Array with 4 colors: red ($R$), green ($G$), blue ($B$) and 
  alpha ($A$), lying between 0 and 255. $A=0$ means totally transparent, and 
  $A = 255$ means totally opaque.
\end{description}

\subsubsection{\texttt{path\_style\_t}}

SVG path style.

\begin{description}
 \item[\texttt{type}] Path type.
 \item[\texttt{from, to}] Path origin and destination.
 \item[\texttt{control}] Control point
 \item[\texttt{width}] Stroke width in pixels.
 \item[\texttt{color}] Array with 4 colors: red ($R$), green ($G$), blue ($B$) and 
  alpha ($A$), lying between 0 and 255. $A=0$ means totally transparent, and 
  $A = 255$ means totally opaque.
\end{description}

 For \texttt{style.type == GRAPH\_STRAIGHT}, draws a straight line from origin
 to destination.
 
 For \texttt{style.type == GRAPH\_PARABOLA}, draws a parabola from origin
 to destination using the control point.
 
 For \texttt{style.type == GRAPH\_CIRCULAR}, draws the arc of a circle from 
 origin to destination using the control point as the circle center.

\subsection{Layout}

\subsubsection{\texttt{graph\_layout\_random}}

Place points uniformly inside specified box.

\begin{description}
 \item[Preconditions]~\\
   \texttt{box} must be a valid box.\\
   \texttt{p} must have dimension $n$.
 \item[Postconditions] \texttt{p[i]} is a random coordinate inside \texttt{box}.
\end{description}

\subsubsection{\texttt{graph\_layout\_random\_wout\_overlap}}

Place points with specified radius uniformly avoiding overlap 
with probability $t$.

\begin{description}
 \item[Preconditions]~\\
   \texttt{radius} must be positive.\\
   $t$ must be a valid probability ($0 \ge t \ge 1$).\\
   \texttt{p} must have dimension $n$.
 \item[Postconditions] \texttt{p[i]} is a random coordinate.
\end{description}

The algorithm determines a box with size $l$ such that, if $n$ 
points with radius $r$ are thrown within it, will not have any 
collision with probability $t$. The formula is derived in 
\href{http://math.stackexchange.com/q/325844/37667}{Math Exchange}.

\begin{equation*}
 l = \frac{nr}{2} \sqrt{\frac{2 \pi}{-\log(1-t)}}
\end{equation*}

\subsection{Printing}

\subsubsection{\texttt{graph\_print\_svg}}

Prints graph as SVG to file, using vertex coordinates given in p and with a 
style for each point and edge.

\begin{description}
 \item[Preconditions]~\\
   \texttt{p} must have dimension $n$.
   \texttt{point\_style} must have dimension $n$.
   \texttt{edge\_style} must have dimension $m$.
 \item[Postconditions]
   \texttt{filename} is a valid SVG file.
\end{description}

Edges are ordered according to vertices' order. In undirected graphs, 
an edge $E_{ij}$ is considered only if $i < j$. In directed graphs,
mutual edges will superimpose if \texttt{edge\_style.type == GRAPH\_STRAIGHT}.

\subsubsection{\texttt{graph\_print\_svg\_one\_style}}

Prints graph as SVG to file, using vertex coordinates given in p and with a 
single style for all points and edges.

\begin{description}
 \item[Preconditions]~\\
   \texttt{p} must have dimension $n$.
 \item[Postconditions]
   \texttt{filename} is a valid SVG file.
\end{description}

The edge style type is ignored, using only \texttt{GRAPH\_STRAIGHT}.

\subsubsection{\texttt{graph\_print\_svg\_some\_styles}}

Prints graph as SVG to file, using vertex coordinates given in \texttt{p} and
with a number of styles given. The mapping vertex$\to$style is given in \texttt{ps},
and the mapping edge$\to$style is given in \texttt{es}.

\begin{description}
 \item[Preconditions]~\\
   \texttt{p} must have dimension $n$.\\
   \texttt{ps} must have dimension $n$.\\
   \texttt{es} must have dimension $m$.\\
   \texttt{point\_style} must have dimension \texttt{num\_point\_style}.\\
   \texttt{edge\_style} must have dimension \texttt{num\_edge\_style}.
 \item[Postconditions]
   \texttt{filename} is a valid SVG file.
\end{description}

This function tries to avoid extensive memory utilization one just some 
styles are desired. If vertex $v_i$ should have style $S_j$, then 
\texttt{ps[i] = j}. Ditto for edges.

Edge order is based on vertices order. In undirected edges, edge $E_{ij}$ 
is considered only if $i < j$.

\end{document}
